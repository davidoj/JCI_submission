%!TEX root = main.tex


\documentclass[USenglish,onecolumn]{article}
\usepackage[big]{dgruyter}

% \usepackage[utf8]{inputenc}%(only for the pdftex engine)
% \RequirePackage[no-math]{fontspec}%(only for the luatex or the xetex engine)
 
\input{Custom_Settings}

  
\begin{document}
 

  \articletype{Research Article{\hfill}Open Access}

  \author*[1]{David Johnston}

  \author[2]{Cheng Soon Ong}

  \author[3]{Robert C. Williamson}

  \affil[1]{Australian National University; E-mail: davidoj@fastmail.com.au}

  \affil[2]{Data61; E-mail; chengsoon.ong@anu.edu.au}

  \affil[3]{Universität Tübingen; Email: bob.williamson@uni-tuebingen.de}

  \title{\huge Causal Inference Without Interventions}

  \runningtitle{Inference Without Interventions}

  %\subtitle{...}

  \begin{abstract}
{Valid causal inferences from diverse data are a tantalizing but elusive prospect. Assumptions are necessary to make any inferences, and these assumptions cannot all be washed out by sufficiently large datasets, and they consequently require strong prior support. We argue that structural interventions in particular frequently embody assumptions not entailed by prior knowledge or by the given data. A key role of structural interventions is to identify fixed relationships that are shared by data arising from observation and by data arising from the consequences of a decision. We show that structural interventions are not required to analyse assumptions of fixed relationships like this. Our starting point is is to ground causal models in the problem of making good decisions, and our first result is an equivalence between decision models with "sequences of identical conditionals" and decision models that feature a symmetry we call "IO contractibility". This can be seen as a generalisation of De Finetti's work on exchangeability, itself an attempt to justify the conventional but seemingly mysterious assumption of sequences featuring a "shared but unknown probability distribution". We argue that IO contractibility is often an unreasonable assumption. Our second result is that IO contractibility may be implied by a combination of an observed conditional independence and a weaker assumption based on the principle of ``precedent'', which requires that everything that might be done has already been done before in some way and has been seen to work. We discuss a connection between this latter result and constraint based causal discovery.}
\end{abstract}
  \keywords{causal inference, decision theory}
   \classification[MSC]{62C99} % Decision theory - other (62C05 also maybe?)
 % \communicated{...}
 % \dedication{...}

  \journalname{Journal of Causal Inference}
\DOI{DOI}
  \startpage{1}
  \received{..}
  \revised{..}
  \accepted{..}

  \journalyear{2022}
  \journalvolume{1}
%  \journalissue{1}
 

\maketitle
\section{Introduction}

Judea Pearl's causal hierarchy distinguishes between three types of problems: prediction problems, intervention problems, and counterfactual problems \citep{pearl_book_2018}. Modelling an intervention problem requires a different kind of knowledge than that required for modelling predictions, and modelling a counterfactual problem requires a different kind of knowledge than that needed for modelling intervention problems.

While we think that Pearl's hierarchy offers an important insight into the differences between causal inference and classical statistics, we feel the terminology of interventions is confusing. In Pearl's theory, \emph{structural interventions} are used to model what he describes as intervention problems. Structural interventions are operations that transform a probability distribution according to a \emph{graphical causal model}. However, ``intervention problems'' refer to a broad class of problems that ask questions like ``what will happen if I do this? what will happen if I do something else instead?'' -- problems that we frequently encounter whether or not we make use of graphical causal models. This kind of problem often comes up we want to make a decision; given various options and some idea of the outcomes we would like to achieve, we want to know what the likely consequences of each option are.

As the terminology suggests, structural interventions and graphical causal models are often used as a tool for modelling the consequences of different options (that is, ``structural interventional models'' are used to model ``intervention problems'' broadly understood). However, using structural interventions to model the consequences of different options poses a conundrum. Our decision maker has some set of options $C$ to consider, and selecting an option $\alpha$ from this set is their decision problem. Given a dataset, they may attempt to infer causal relationships using causal discovery, or to estimate certain structural intervention-based effects of interest from the data along with whatever causal assumptions they supplied at the outset. Typically, whatever learning or estimation takes place, it does \emph{not} include an attempt to learn a correspondence between each of their options $\alpha$ and a structural intervention -- this correspondence is usually assumed to be prior knowledge. However, such a correspondence seemingly goes beyond the kind of prior knowledge a decision maker is likely to have.

The most common kind of structural intervention is known as a \emph{perfect} or \emph{hard} intervention. A perfect intervention is often denoted with the symbol $\mathrm{do}(\RV{X}=x)$. Given a probability distribution $\prob{P}$, a variable $\RV{X}$ and a causal graphical model $\mathcal{G}$ which, among other things, specifies a set of \emph{causal parents} $\mathrm{Pa}(\RV{X})$ of $\RV{X}$, the intervention $\mathrm{do}(\RV{X}=x)$ yields a new probability distribution $\prob{P}'$ such that the conditional probability $\prob{P}^{\prime \RV{X}|\mathrm{Pa}(\RV{X})}$ becomes the function $\cdot \mapsto \delta_x$, while all other conditional distributions of a ``child'' conditional on its ``parents'' according to $\mathcal{G}$ match their counterparts in $\prob{P}$ \citep[Sec. 1.3.1]{pearl_causality:_2009}

Thus identifying a perfect intervention $\mathrm{do}(\RV{X}=x)$ with an option $\alpha$ embodies a collection of assumptions -- first, it embodies the assumption that selecting the option $\alpha$ will force future observations of the variable $\RV{X}$ to take the value $x$. This information may sometimes (though not always) be available to decision makers. However, the identification of options with perfect interventions also embodies the assumption that selecting the option $\alpha$ will leave all ``parental conditionals'' with respect to $\mathcal{G}$ unchanged with the exception of $\prob{P}^{\prime \RV{X}|\mathrm{Pa}(\RV{X})}$. It's harder to see how a decision maker could know this.

It's possible that the decision maker might choose the graph $\mathcal{G}$ carefully just so that these additional conditions hold. If so, the decision maker must know exactly which conditionals are invariant a priori, and the graphical model $\mathcal{G}$ is merely a convenient shorthand for representing this knowledge. Typically a decision maker's prior knowledge will not be so extensive. Furthermore, adapting the graph to the set of options under consideration conflicts with normal practice in causal discovery where the learned graph does not depend on the options under consideration. Common measures of success in causal discovery are the structural intervention distance \citep{peters_structural_2015} and the structural hamming distance, neither of which depend on the options under consideration \citep{scherrer_learning_2022, toth_active_2022, brouillard_differentiable_2020, forre_constraint-based_2018, chickering_optimal_2003}. The idea that the relationships captured by a causal graph are independent of the options under consideration is also defended in \citet{pearl_does_2018}. 

% I don't know why these citations don't work
% \citep{ngGraphAutoencoderApproach2019 ,zhengDAGsNOTEARS2018, spirtes_causation_1993}

On the other hand, it is not obvious to us how a decision maker could know a priori that their options correspond to perfect interventions on some unknown ``objectively correct'' graphical model. We can also raise a concrete arguments against the notion that actions known in advance to affect a certain variable can typically be assumed to be interventions on that variable: there are multiple different ways to influence many variables we are interested in measuring and they cannot all be perfect interventions. \citet{hernan_does_2008,noauthor_does_2016} considers the example of different options that are known a priori to affect a person's body mass index, including diet plans, gastric bypass surgery and limb removal. These will all plausibly affect an individual's risk of death differently, and so they cannot all be modeled by the same intervention on body mass index. Further, it seems to us (as well as other authors in this exchange \citep{pearl_does_2018,hernanInvitedCommentaryCausal2009,shahar_association_2009}) that none of these options stand out as a strong candidate to be identified with a perfect intervention on body mass index. The difficulty is twofold: first, we don't know what makes the ``true graph'' true, and as a result it is not clear how to decide which option if any should be be identified with perfect interventions on this graph. Second, it is not even obvious which action represents the ``canonical'' way to alter a person's body mass index so it is not clear how we could verify that any procedure for learning a causal graph actually yields the correct interventions as a result.

Hernán argues that limb removal can be dismissed as an option because it is not interesting from a scientific or public health standpoint. While this is a reasonable contention, demanding that a causal graph correctly represent options that are interesting from a scientific or public health standpoint is a version of the strategy of of choosing the graph so that the options of interest are correctly represented, which as we've mentioned is not the standard practice in causal discovery.

Perhaps in recognition of the fact that many actions are not modeled by perfect interventions, there is a diverse array of structural interventions that can be found in the literature: a non-exhaustive review reveals perfect interventions or ``hard'' interventions \citep[ch. ~1]{pearl_causality:_2009,hauser_characterization_2012}, soft interventions \citep{correa_calculus_2020,eberhardt_interventions_2007}, general or fat-hand interventions \citep{eberhardt_interventions_2007,yang_characterizing_2018,glymour_evaluating_2017} and general interventions with unknown targets \citep{brouillard_differentiable_2020}. However, as in the case of perfect interventions, none of these generalised families of structural interventions make provisions for learning the option-intervention correspondence.

Notwithstanding our criticisms, the assumptions of invariant parental conditional distributions made by structural interventional models play a very important role in the interpretation of graphical causal models. If a decision maker wants to learn from some observed data so as to make a better decision, then these assumptions tell them what the observational data and the consequences of their decisions will have in common: the parental conditional distributions. In this paper, we show that structural interventions are not the \emph{only} way of analysing assumptions of this type.

We investigate whether certain assumptions of symmetry are a viable alternative to structured interventions for this task of using features of observations to forecast consequences of different options. In particular, we are interested in assumptions about the future being like the past. In probability models, an assumption of this type is the assumption of \emph{exchangeability}, which holds that reordering a sequence of observed variables leaves a forecaster with exactly the same prediction problem. This assumption cannot be applied as-is to decision models because, in decision problems, future events are affected by the choices made by the decision maker and therefore cannot be in all respects similar to previously observed events.

We investigate two different assumptions inspired by the idea that ``the future is in some sense like the past''. First, we introduce the idea of \emph{input-output contractibility} (IO contractibility), which can be viewed as a generalisation of exchangeability to decision models. In particular, we show a theorem analogous to De Finetti's famous representation \citep{de_finetti_foresight_1992} theorem holds; IO contractibility is equivalent to the assumption that there is a shared but unknown input-output map for both the observations and the consequences of a decision maker's choices (Theorem \ref{th:ciid_rep_kernel}). Unlike exchangeability, however, IO contractibility is not an appealing assumption in many data-driven decision problems.

We subsequently explore the general principle of \emph{precedent}, which holds that everything a decision maker can do has been done before.  We show that under the assumption of \emph{diverse precedent}, based on the principle of precedent, conditional independences imply the stronger conclusion of IO contractibility and with it the possibility of estimating an input-output conditional distribution from the given observational data (Theorem \ref{th:latent_to_observable}). A key assumption of Theorem \ref{th:latent_to_observable} is that the posterior of the parametrisation of a certain conditional distribution is dominated by the uniform measure. We discuss, speculatively, how this assumption may be related to causal structures, noting similar assumptions that appear in \citet{meek_strong_1995} and \citet{janzingCausalVersionsMaximum2021}.

Section \ref{sec:tech_prereq} introduces the formalism of decision models. These differ from probability models in that they are a map from a set of options to distributions over consequences. We make use of the notion of \emph{extended conditional independence} introduced by \citet{constantinou_extended_2017}, which is a notion of conditional independence relevant to decision models. Section \ref{sec:evaluating_decisions} introduces the idea of shared conditionally independent and identical responses, and shows that this is equivalent to the assumption of input-output in Theorem \ref{th:ciid_rep_kernel}. Section \ref{sec:precedent} explains the assumption of precedent and proves Theorem \ref{th:latent_to_observable} and discusses the interpretation of this assumption and its connection to structural causal models.

\subsection{Previous work on symmetries in causal inference}\label{sec:prev_work}

The approach that we take assumes that decision making is the fundamental problem that requires causal inference. This assumption motivates the formalism of ``decision models'' that we use to investigate the questions raised here. The broad idea of starting with the options available to a decision maker rather than starting with some foundational notion of causation is often called the \emph{decision theoretic approach to causal inference} \citep{heckerman_decision-theoretic_1995,dawid_decision-theoretic_2012,dawid_decision-theoretic_2020}. \citet{lattimore_causal_2019,lattimore_replacing_2019} also document an approach to causal modelling that demands explicit consideration of the set of interventions, and is arguably an example of the decision theoretic approach.

\citet{lindley_role_1981} discussed sequences of exchangeable observations along with ``one more observation''. Lindley mentioned the application of this model to questions of causation, but did not explore this deeply due to the perceived difficulty of finding a satisfactory definition of causation. \citet{rubin_causal_2005, imbens_causal_2015} made use of the assumption of models with exchangeable potential outcomes to prove several identification results. \citet{saarela_role_2020}, used graphical causal models to propose \emph{conditional exchangeability}, defined as the exchangeability of the non-intervened causal parents of a target variable under intervention on its remaining parents. Sareela et. al. suggested that this could be interpreted as a symmetry of an experiment involving administering treatments to patients with respect to exchanging the patients in the experiment. \citet{hernan_estimating_2006,hernan_beyond_2012,greenland_identifiability_1986,banerjee_chapter_2017,dawid_decision-theoretic_2020} all discuss similar experimental symmetries. These symmetries are reminiscent of \emph{exchange commutativity} discussed here. They're not identical, however -- exchange commutativity can be justified by the equivalence of certain prediction problems that arise from a single experiment, instead of an equivalence of different experiments that arise from, for example, interchanging experimental subjects.

A different kind of regularity of causal models is given by the stable unit treatment distribution assumption (SUTDA) in \citet{dawid_decision-theoretic_2020} and the stable unit treatment value assumption (SUTVA) in \citep{rubin_causal_2005}. This regularity is similar to the condition of \emph{locality} introduced here.

Theorem \ref{th:latent_to_observable} was inspired by the idea of \emph{causal inference by invariant prediction} \citep{peters_causal_2016}. While both the assumptions and the conclusions drawn in that work differ from the assumptions and conclusion of Theorem \ref{th:latent_to_observable}, both proceed from an idea that can be roughly described as ``things I can do have been done before'' and both look for variable pairs $\RV{X}$ and $\RV{Y}$ such that the distribution of $\RV{Y}$ given $\RV{X}$ doesn't change after actions are taken. Finally, the variable described in that work as ``the environment'' is similar to the variable $\RV{Z}$ in Theorem \ref{th:latent_to_observable} in that neither variable needs to be IID, and both variables are only necessarily of interest in the observation set, and need not be of any interest for the consequences of actions.

\input{technical}
\input{contractibility}
%!TEX root = main.tex

\section{Precedent}\label{sec:precedent}

We have suggested that IO contractibility is usually an unreasonably strong assumption for a decision maker to make, on the grounds that it implies overly strong interchangeability properties between different datasets. One way to get around this objection is to suppose that conditionally independent and identical responses are shared by pairs $(\RV{E}_i,\RV{X}_{i})$ where the $\RV{E}_i$ are in fact latent variables. In this case, the assumption would still assert that infinite $(\RV{E}_i,\RV{X}_{i})$ sequences arising from observation would be interchangeable with infinite $(\RV{E}_j,\RV{X}_{j})$ sequences arising as consequences of actions, but because the $\RV{E}_i$ are never observed these interchanges do not imply that we would use the same model for different experiments.

To understand this construction, we will consider a kind of decision model featuring long sequence of exchangeable observations indexed by natural numbers and ``one more'' variable representing the ``consequences of action'' indexed by the special character $c$. That is, we have $(\RV{E}_i,\RV{X}_i)_{i\in \mathbb{N}}$ unresponsive to the decision maker's choice and $(\RV{E}_c,\RV{X}_c)$ responsive to this choice. Call this setup a ``see-do model''.

We can relate the minimum size of the set $E$ of possible values of the latent inputs to the number of different options available to the decision maker. In particular it is always possible to construct a see-do model with latent conditionally independent and identical responses $\RV{E}_i$ from a see-do model $(\prob{P}_\cdot,\Omega,\sigalg{F})$ of $(\RV{X}_i)_{i\in \mathbb{N}\cup\{c\}}$ where the range $E$ of the variables $\RV{E}_i$ has size at least equal to the number of linearly independent options and observations.

\begin{definition}[Dimension]
Given a collection of probability distributions $A=\{\prob{P}_i|i\in B\}$ on a discrete space $X$, let $p_i:=(\prob{P}_i(\{x\}))_{x\in X}$. Then $\mathrm{dim}(A)=\mathrm{dim}(\mathrm{span}(\{p_i|i\in B\}))$.
\end{definition}

\begin{theorem}[construction of latent inputs]\label{th:construction_latent_inputs}
Suppose a decision model $(\prob{P}_\cdot,C,\Omega)$ and observable variables $\RV{X}:=(\RV{X}_i)_{i\in \mathbb{N}\cup\{c\}}$ with $X$ discrete, $\RV{X}_{\mathbb{N}}$ exchangeable, $\RV{X}_\mathbb{N}\CI^e\mathrm{id}_C$ and $\RV{X}_c\CI^e\RV{X}_\mathbb{N}|(\RV{G},\mathrm{id}_C)$ where $\RV{G}$ is the directing random measure of $\RV{X}_\mathbb{N}$. Let 
\begin{align}
A_g:=\{\prob{P}_C^{\RV{X}_1\RV{G}}(\cdot|g)\}\cup\{\prob{P}_\alpha^{\RV{X}_c|\RV{G}}(\cdot|g)|\alpha\in C\}
\end{align}
and take $A:=\argmax_{\{A_g|g\in \Delta(X)\}}(\mathrm{dim}(A_g))$; assume $A$ is a finite set.

Then there exists a sequence $\RV{E}:=(\RV{E}_i)_{i\in \mathbb{N}\cup\{c\}}$ on a refinement $\Omega'$ of $\Omega$ with $|E|= \mathrm{dim}(A)$ such that $(\prob{P}_\cdot',\RV{E},\RV{X})$ is IO contractible and for all $\alpha$, $\prob{P}_\alpha^{\prime \RV{X}} = \prob{P}_\alpha^{\RV{X}}$.

Moreover, for any such sequence, $|E|\geq \mathrm{dim}(A)$.
\end{theorem}

\begin{proof}
See Appendix \ref{sec:proof_precedent}
\end{proof}

\begin{example}
More concretely, suppose we have an infinite set of observations $(\RV{X}_i)_{i\in \mathbb{N}}$ and one ``consequence'' $\RV{X}_c$. $X$ is binary, and the control we can exert is to choose either $\prob{P}_{0}^{\RV{X}} = \frac{1}{4}\delta_0 + \frac{3}{4}\delta_1$ or $\prob{P}_{1}^{\RV{X}} = \frac{3}{4}\delta_0 + \frac{1}{4}\delta_1$, independent of all other observations. Suppose further that for $i\in \mathbb{N}$, $\prob{P}_\alpha^{\RV{X}_i}=\delta_1$ independent of all other observations for all $\alpha\in\{0,1\}$. Then, because the dimension of
\begin{align}
	A=\{\frac{1}{4}\delta_0 + \frac{3}{4}\delta_1,\frac{3}{4}\delta_0 + \frac{1}{4}\delta_1,\delta_1\}
\end{align}
is 2, we can consider this model to be IO contractible with inputs $\RV{E}_i\in\{0,1\}$ such that
\begin{align}
	\prob{P}_\alpha^{\RV{X}_i|\RV{E}_i}(\cdot|e) &= \delta_e
\end{align} 
in this simple example, the directing measure $\RV{G}$ and the directing conditional $\RV{H}$ are trivial.
\end{example}

A special case of a see-do model with conditionally independent and identical responses is when, among the observations, $\prob{P}_\alpha^{\RV{E}_1|\RV{G}}(\cdot|g)$ has full support almost surely. In such a case, roughly speaking, the consequences of anything the decision maker can do have already been seen. We refer to this as a model in which the decision maker's actions have \emph{precedent}.

Theorem \ref{th:latent_to_observable} shows that a slightly strengthened version of this assumption of precedent can have signigicant implications for a decision maker who wants to infer consequences from their observations. This theorem is motivated by the following example:

\begin{example}\label{ex:doctor_precedent}
Suppose we have a collection of doctors who each see a series of patients, offer a treatment $\RV{X}_i$ and report their results $\RV{Y}_i$. Each doctor may decide whether or not to prescribe based on any number of unobserved factors, and may offer additional unrecorded treatments, vary in their bedside manner and so forth, and these decisions could be stochastic. The decision maker is \emph{also} a doctor, and is reviewing the data contained in the sequences $(\RV{Z}_i,\RV{X}_i,\RV{Y}_i)_{i\in [n]}$, where $\RV{Z}_i$ identifies the doctor involved in the $i$th treatment interaction. The decision maker supposes that whatever overall treatment plan they will adopt (which could and probably does also involve features not listed in this set of variables), the same plan has probably been executed at least sometimes by some of these other doctors. Because the other doctors have some variation in their treatment behaviour, it stands to reason that different doctors making the same prescription decisions should see different results \emph{if, conditional on the prescription, the different treatment plans actually lead to different results}. Conversely, if there is \emph{no} variation in results different doctors obtain, then whether or not treatment occurred is presumably the \emph{only} important feature of any treatment plan.

This story might fail if the doctors all knew exactly the long-run probabilistic outcomes of different treatment plans and coordinated with one another to mask any variation they induced. For example, doctor 1 picks a medium effectiveness unobserved plan 100\% of the time, while doctor 2 picks a highly effective unobserved plan 50\% of the time and a low effectiveness unobserved plan 50\% of the time, leading to the same distribution over outcomes. Approximate coordination is plausible -- everyone is likely to be aiming for similar goals, and may therefore make choices that are similarly effective. In order to conclude that the lack of variation between doctors is indicative of the importance of prescription decisions, our decision maker must somehow rule out coordination of this kind.

Note that $\RV{X}_i$ needn't be limited to a particular treatment; in principle, the decision maker might explore many different candidates for a variable $\RV{X}_i$ which renders $\RV{Y}_i$ conditionally independent of $\RV{Z}_i$.
\end{example}

Theorem \ref{th:latent_to_observable} establishes formal conditions for the informal deduction described in Example \ref{ex:doctor_precedent}. 

\begin{definition}[Index notation for discrete conditionals]
Given a joint probability distribution $\mu^{\RV{XY}}$ with $\RV{X}$ and $\RV{Y}$ discrete, let $\mu^y_x:=\mu^{\RV{Y}|\RV{X}}(\{y\}|x)$ and $\mu^Y_X:= (x,y)\mapsto \mu^y_x$
\end{definition}

\begin{theorem}[Latent to observable IO contractibility]\label{th:latent_to_observable}
Given a decision model $(\prob{P}_\cdot,(C,\sigalg{C}),(\Omega,\sigalg{F})$ and sequences $(\RV{E}_i,\RV{X}_i,\RV{Y}_i)_{i\in\mathbb{N}\cup\{c\}}$, $(\RV{Z}_i)_{i\in \mathbb{N}}$ all taking values in discrete sets, suppose among the observations $i\in \mathbb{N}$, the pairs $(\RV{E}_i,(\RV{X}_i,\RV{Y}_i,\RV{Z}_i))$ share conditionally independent and identical responses and for all $i\in \mathbb{N}\cup\{c\}$ pairs $(\RV{E}_i,(\RV{X}_i,\RV{Y}_i))$ share conditionally independent and identical responses. Take $\RV{G}$ to be the directing random conditional of $(\prob{P}_\cdot,\RV{E}_{\mathbb{N}},(\RV{X}_i,\RV{Y}_i,\RV{Z}_i)_{i\in \mathbb{N}})$ and $\RV{H}$ to be the directing random conditional of $(\prob{P}_\cdot, \RV{E}_{\mathbb{N}\cup\{c\}}, (\RV{X}_i,\RV{Y}_i)_{i\in \mathbb{N}\cup\{c\}})$. 

Let $I\subset \Delta(Y)^{XZ}$ be the event $\RV{G}^Y_{Xz}=\RV{G}^Y_{Xz'}$ for all $z,z'\in Z$; i.e. the event that $\RV{Y}_i$ is independent of $\RV{Z}_i$ conditional on $\RV{X}_i$ and $\RV{G}$. For arbitrary $\alpha$, $\prob{Q}_\alpha\in \Delta(\Omega)$ be the probability measure such that, for all $A\in \sigalg{F}$
\begin{align}
\prob{Q}_\alpha(A) := \prob{P}_\alpha^{\mathrm{id}_\Omega|\mathds{1}_I\circ \RV{G}^{Y}_{XZ}}(A|1)
\end{align}
i.e. $\prob{Q}_\alpha$ is $\prob{P}_\alpha$ conditioned on $\RV{G}^{Y}_{XZ}\in I$.

Thus $\RV{Y}_i \CI^e_{\prob{Q}} \RV{Z}_i | (\RV{X}_i, \mathrm{id}_C)$. Suppos for all $\alpha$, $\prob{Q}_\alpha$-almost all $z,z'\in Z$, $e\in E$, $g^E_{z}\in \Delta(E)$, $g^{XY}_{EZ}\in \Delta(X\times Y)^{E\times Z}$, $\prob{Q}_\alpha$ satisfies the \emph{dominated posterior} assumption:
\begin{align}
	\prob{Q}_{\alpha}^{\RV{G}^E_{z'}|\RV{G}^{XY}_{EZ}\RV{G}^E_{z}}(\cdot|g^{XY}_{EZ},g^e_{z}) \ll U_{\Delta(D)}\label{eq:lebesgue_dom}
\end{align}
Where $U_{\Delta(D)}$ is the uniform measure on the $|D-1|$ simplex of discrete probability distributions with $|D|$ outcomes. Then $(\prob{Q}_\cdot,\RV{X},\RV{Y})$ is also IO contractible.
\end{theorem}

\begin{proof}
We show that the assumption of conditional independence imposes a polynomial constraint on $\RV{G}^d_z$ which is nontrivial unless $\RV{Y}_i\CI^e (\RV{Z}_i,\RV{E}_i,\text{id}_C)|(\RV{X}_i,\RV{H})$, and hence the solution set $S$ for this constraint has measure 0 when this conditional independence does not hold.

Full proof in Appendix \ref{sec:proof_precedent}.
\end{proof}

\section{Under what circumstances are latent IO contractible models appropriate?}

The crucial assumption in Theorem \ref{th:latent_to_observable} -- apart from latent IO contractibility -- is the assumption that the distribution of the conditional distribution of the latent variable is dominated by the Lebesgue measure. To see why this is critical, consider that every sequence $(\RV{X}_i,\RV{Y}_i)_{i\in\mathbb{N}}$ can be transformed to the IO contractible sequence $((\RV{X}_i,\RV{Y}_i),(\RV{X}_i,\RV{Y}_i))_{i\in\mathbb{N}}$. Thus, were the dominated posterior assumption not required by Theorem \ref{th:latent_to_observable}, \emph{any} nontrivial conditional independence would imply observable IO contractibility. However, the sequence $((\RV{X}_i,\RV{Y}_i),(\RV{X}_i,\RV{Y}_i))_{i\in\mathbb{N}}$ does not satisfy the dominated posterior assumption. In particular, if $\RV{Y}_i \CI^e_{\prob{Q}} \RV{Z}_i | (\RV{X}_i,\RV{G}^Y_{XZ}, \mathrm{id}_C)$ then fixing $\RV{G}^{XY}_{z}=g^{XY}_z$ for some $z$ implies $\RV{G}^{XY}_{z'}$ must be such that $\RV{G}^Y_{Xz} = \RV{G}^Y_{Xz'}$, a Lebesgue measure 0 event.

If $\prob{P}_\alpha^{\RV{G}}$ is dominated by the uniform measure on $\Delta(EXYZ)$, then $\prob{P}_\alpha^{\RV{G}^E_Z|\RV{G}^{XY}_{EZ}}(\cdot|g^{XY}_{EZ})$ is dominated by the uniform measure on $\Delta(E)^Z$ for almost all $(g^{XY}_{EZ},g^Z)$ \citep[pg. 155]{cinlar_probability_2011}. However, this is not enough for Theorem \ref{th:latent_to_observable} -- we condition on $I\subset \Delta(Y)^{XZ}$, which is a measure 0 event with respect to the uniform measure on $\Delta(EXYZ)$.

In light of this, it would be very useful to extend Theorem \ref{th:latent_to_observable} to an approximate result. Specifically, in the event $\RV{Y}_i$ is approximately independent of $\RV{Z}_i$ given $\RV{X}_i$ and $\RV{G}$, under what conditions is $\RV{Y}_i$ also approximately independent of $\RV{E}_i$ given $\RV{X}_i$ and $\RV{G}$? 

For theorem \ref{th:latent_to_observable} to hold, the latent inputs must support the assumption of a dominated posterior for the conditional $\RV{G}^E_Z$, and for an approximate result along the same lines we posit that a stronger requirement of diversity for the posterior over conditional distributions $\{\RV{G}^E_z|z\in Z\}$ will be necessary. We don't know in general how these requirements should be understood.

The dominated posterior assumption also has a connection to the theory of causal graphical models. \citet{meek_strong_1995} justified the \emph{faithfulness} condition for causal graphs associated with discrete probability models on the assumption that the distribution of parameters of a distribution consistent with a particular causal graph are dominated by the Lebesgue measure. In this theory, we have a discrete set of hypotheses over causal structures that imply some conditional independences, and Lebesge-dominated priors over the directing measure after conditioning on any of the causal structure hypotheses and their associated independences. Applying similar reasoning to the present case, we posit an argument along these lines: if we have the independence $\RV{Y}_i\CI^e_{\prob{Q}}(\RV{E}_i,\RV{Z}_i)|(\RV{X}_i,\RV{G},\mathrm{id}_C)$ but not the independence $\RV{E}_i\CI^e_{\prob{Q}} \RV{Z}_i|(\RV{G},\mathrm{id}_C)$ and furthermore $\RV{Z}_i$ is an ancestor of $\RV{E}_i$ and $(\RV{E}_i,\RV{Z}_i)$ is an ancestor of $(\RV{X}_i,\RV{Y}_i)$ (so that $\RV{G}^E_Z$ and $\RV{G}^{XY}_{EZ}$ are associated with forward edges in the causal model) then the dominated posterior assumption may be supported. Note that it may be possible to rule out the independence $\RV{E}_i\CI^e_{\prob{Q}} \RV{Z}_i|(\RV{G},\mathrm{id}_C)$ on the basis of the non-independence of $\RV{Z}_i$ and $\RV{X}_i$.

Another relation between theory of causal graphical models and the present work may be found in the \emph{causal version of the principle of maximum entropy} \citep{sunCausalInferenceChoosing2006,janzingCausalVersionsMaximum2021}. The causal version of the principle of maximum entropy, in contrast to the standard version of the principle, suggests that priors be specified by sequentially maximising the entropy of a cause, then maximising the conditional entropy of the first effect given the cause and so forth. While the cited articles discuss using the principle of entropy maximisation to specify prior distributions over observed variables rather than distributions over directing conditionals, the same principle may perhaps be applied to the specification of priors over directing conditionals. We posit that the causal version of the prinicple of maximum entropy might support a similar line of argument: if $\RV{Y}_i\CI^e_{\prob{Q}}(\RV{E}_i,\RV{Z}_i)|(\RV{X}_i,\RV{G},\mathrm{id}_C)$ but not independence $\RV{E}_i\CI^e_{\prob{Q}} \RV{Z}_i|(\RV{G},\mathrm{id}_C)$ and $\RV{Z}_i$ is an ancestor of $\RV{E}_i$ and $(\RV{E}_i,\RV{Z}_i)$ is an ancestor of $(\RV{X}_i,\RV{Y}_i)$, then perhaps the causal version of the principle of maximum entropy offers some support for the dominated posterior assumption. Note that this (as well as the implication suggested in the previous paragraph) are highly speculative.

\section{Conclusion}

We employ a decision theoretic approach to causal inference to investigate two different approaches to answering the question ``how do my observations relate to the consequences of my choices?'' Our approach allows us to consider the question of what observations and consequences have in common independently from any prior knowledge the decision maker might have about how their choices influence outcomes -- neither Theorem \ref{th:ciid_rep_kernel} nor Theorem \ref{th:latent_to_observable} depend on any assumptions about a decision maker's prior knowledge of the effects of their different options (though the plausibility of the assumptions in both theorems may well depend on such prior knowledge).

The grand aim of this work is to facilitate causal inference in situations where a decision maker has relatively little causal knowledge at the outset. We think avoiding structured interventions in this setting is advantageous because we regard the question of whether an action is known in advance to influence a particular variable as substantially more transparent than the question of whether it is well modeled by a structured intervention (of any type) on that variable.

Nevertheless, this work leaves many open questions for causal inference in the low prior knowledge setting. We have argued that the assumptions required for Theorem \ref{th:ciid_rep_kernel} are unlikely to be compelling in many situations. While Theorem \ref{th:latent_to_observable} may be more broadly plausible, we've identified the ``dominated posterior'' assumption as a particularly difficult one to evaluate. We've suggested that there might be a connection between this assumption and causal structure assumptions. If this is so, one might also want to ask how often the relevant structural assumptions are transparent to a decision maker.

For any practical inference, a generalisation of Theorem \ref{th:latent_to_observable} to approximate independence is in order. Such a generalisation may bring additional clarity to the dominated posterior assumption.

Despite these challenges, we are encouraged by a number of features of this work. Using decision making as a starting point for constructing models means that, at the outset, we are only making commitments a decision maker is likely to already be making if they want to apply a formal theory of decision making. The informal idea of precedent that underpins Theorem \ref{th:latent_to_observable} seems like a general principle that may be applicable in a broad range of data-driven decision making problems. Finally, the apparent connection between Theorem \ref{th:latent_to_observable} suggests that much of the work already done in the world of causal graphical models may be applicable to our alternative perspective. Causal inference under circumstances of limited prior knowledge presents many hard conceptual as well as practical problems, and our approach is a promising new avenue of investigation.

\bibliographystyle{plainnat}

\bibliography{library}

\appendix

\input{appendix}

\end{document}