%!TEX root = main.tex


\section{Technical Prerequisites}\label{sec:tech_prereq}

Our approach to causal inference is based on probability theory. Many results and conventions will be familiar to readers, and these are collected in Section \ref{sec:standard_prob}. Because decision models are stochastic functions rather than probability measures (Section \ref{sec:probability_sets}), we make use a generalisation of conditional independence called \emph{extended conditional independence}, explained in Section \ref{sec:eci}.

\subsection{Probability Theory}\label{sec:standard_prob}

\subsubsection{Measurable spaces}

\begin{definition}[Sigma algebra]
Given a set $A$, a $\sigma$-algebra $\mathcal{A}$ is a collection of subsets of $A$ where
\begin{itemize}
	\item $A\in \mathcal{A}$ and $\emptyset\in \mathcal{A}$
	\item $B\in \mathcal{A}\implies B^{\complement}\in\mathcal{A}$
	\item $\mathcal{A}$ is closed under countable unions: For any countable collection $\{B_i|i\in Z\subset \mathbb{N}\}$ of elements of $\mathcal{A}$, $\cup_{i\in Z}B_i\in \mathcal{A}$ 
\end{itemize}
\end{definition}

\begin{definition}[Measurable space]
A measurable space $(A,\mathcal{A})$ is a set $A$ along with a $\sigma$-algebra $\mathcal{A}$.
\end{definition}

\begin{definition}[Sigma algebra generated by a set]
Given a set $A$ and an arbitrary collection of subsets $U\supset\mathscr{P}(A)$, the $\sigma$-algebra generated by $U$, $\sigma(U)$, is the smallest $\sigma$-algebra containing $U$.
\end{definition}

\paragraph{Common $\sigma$ algebras}

For any $A$, $\{\emptyset,A\}$ is a $\sigma$-algebra. In particular, it is the only sigma algebra for any one element set $\{*\}$.

For countable $A$, the power set $\mathscr{P}(A)$ is known as the discrete $\sigma$-algebra.

Given $A$ and a collection of subsets of $B\subset\mathscr{P}(A)$, $\sigma(B)$ is the smallest $\sigma$-algebra containing all the elements of $B$. 

If $A$ is a topological space with open sets $T$, $\mathcal{B}(\mathbb{R}):=\sigma(T)$ is the \emph{Borel $\sigma$-algebra} on $A$.

If $A$ is a separable, completely metrizable topological space, then $(A,\mathcal{B}(A))$ is a \emph{standard measurable set}. All standard measurable sets are isomorphic to either $(\mathbb{R},B(\mathbb{R}))$ or $(C,\mathscr{P}(C))$ for denumerable $C$ \citep[Chap. 1]{cinlar_probability_2011}.

\subsubsection{Probability measures and Markov kernels}

\begin{definition}[Probability measure]\label{def:prob_meas}
Given a measurable space $(E,\sigalg{E})$, a map $\mu:\sigalg{E}\to [0,1]$ is a \emph{probability measure} if
\begin{itemize}
	\item $\mu(E)=1$, $\mu(\emptyset)=0$
	\item Given countable collection $\{A_i\}\subset\mathscr{E}$, $\mu(\cup_{i} A_i) = \sum_i \mu(A_i)$
\end{itemize}
\end{definition}

\begin{definition}[Set of all probability measures]\label{no:prob_meas_set}
The set of all probability measures on $(E,\sigalg{E})$ is written $\Delta(E)$. We equip $\Delta(E)$ with the coarsest $\sigma$-algebra such that the evaluation maps $\eta_B:\nu\mapsto \nu(B)$ are measurable for all $B\in \sigalg{F}$.
\end{definition}

\begin{definition}[Probability space]
A probability space is a triple $(\mu,E,\sigalg{E})$ consisting of a probability measure and a measurable space.
\end{definition}

\begin{definition}[Markov kernel]\label{def:markov_kern}
Given measurable spaces $(E,\sigalg{E})$ and $(F,\sigalg{F})$, a \emph{Markov kernel} or \emph{stochastic function} is a map $\kernel{M}:E\times\sigalg{F}\to [0,1]$ such that
\begin{itemize}
	\item The map $\kernel{M}(A|\cdot):x\mapsto \kernel{M}(A|x)$ is $\sigalg{E}$-measurable for all $A\in \sigalg{F}$
	\item The map $\kernel{M}(\cdot|x):A\mapsto \kernel{M}(A|x)$ is a probability measure on $(F,\sigalg{F})$ for all $x\in E$
\end{itemize}
\end{definition}

\begin{notation}[Signature of a Markov kernel]
Given measurable spaces $(E,\sigalg{E})$ and $(F,\sigalg{F})$ and $\kernel{M}:E\times\sigalg{F}\to [0,1]$, we write the signature of $\kernel{M}:E\kto F$, read ``$\kernel{M}$ maps from $E$ to probability measures on $F$''.
\end{notation}

\begin{definition}[Deterministic Markov kernel]
A \emph{deterministic} Markov kernel $\kernel{A}:E\to \Delta(\mathcal{F})$ is a kernel such that $\kernel{A}_x(B)\in\{0,1\}$ for all $x\in E$, $B\in\mathcal{F}$.
\end{definition}

\paragraph{Common probability measures and Markov kernels}

\begin{definition}[Dirac measure]\label{def:dirac_meas}
The \emph{Dirac measure} $\delta_x\in \Delta(X)$ is a probability measure such that $\delta_x(A)=\llbracket x\in A \rrbracket$
\end{definition}

\begin{definition}[Markov kernel associated with a function]\label{def:mkern_func}
Given measurable $f:(X,\sigalg{X})\to (Y,\sigalg{Y})$, $\kernel{F}_f:X\kto Y$ is the Markov kernel given by $x\mapsto \delta_{f(x)}$
\end{definition}

\begin{definition}[Markov kernel associated with a probability measure]
Given $(X,\sigalg{X})$, a one-element measurable space $(\{*\},\{\{*\},\emptyset\})$ and a probability measure $\mu\in \Delta(X)$, the associated Markov kernel $\kernel{Q}_\mu:\{*\}\kto X$ is the unique Markov kernel $*\mapsto \mu$
\end{definition}

\subsubsection{Variables, conditionals and marginals}

\begin{definition}[Random variable]\label{def:variable}
Given a measurable space $(\Omega,\sigalg{F})$, which we refer to as a \emph{sample space}, and a measurable space of values $(X,\sigalg{X})$, an \emph{$X$-valued random variable on $\Omega$} is a measurable function $\RV{X}:(\Omega,\sigalg{F})\to (X,\sigalg{X})$.
\end{definition}

A sequence of random variables is also a random variable.

\begin{definition}[Sequence of variables]\label{def:seqvar}
Given a sample space $(\Omega,\sigalg{F})$ and two random variables $\RV{X}:(\Omega,\sigalg{F})\to (X,\sigalg{X})$, $\RV{Y}:(\Omega,\sigalg{F})\to (Y,\sigalg{Y})$, $(\RV{X},\RV{Y}):\Omega\to X\times Y$ is the random variable $\omega\mapsto (\RV{X}(\omega),\RV{Y}(\omega))$.
\end{definition}

We define a partial order on random variables such that $\RV{Y}$ is higher than $\RV{X}$ if $\RV{X}$ is given by application of a function to $\RV{Y}$. For example, $\RV{Y}\varlessthan (\RV{W},\RV{Y})$ as $\RV{Y}$ can be obtained by composing a projection with $(\RV{W},\RV{Y})$.

\begin{definition}[Random variables determined by another random variable]\label{def:variable_po}
Given a sample space $(\Omega,\sigalg{F})$ and variables $\RV{X}:\Omega\to X$, $\RV{Y}:\Omega\to Y$, $\RV{X}\varlessthan \RV{Y}$ if there is some $f:Y\to X$ such that $\RV{X}=f\circ \RV{Y}$.
\end{definition}

We use superscripts to specify marginal and conditional distributions, as subscrips (which are a somewhat more common notation) are reserved for specifying options in decision models (Section \ref{sec:probability_sets}).

\begin{definition}[Marginal distribution]\label{def:pushforward}
Given a probability space $(\mu,\Omega,\sigalg{F})$ and a variable $\RV{X}:\Omega\to (X,\sigalg{X})$, the \emph{marginal distribution} of $\RV{X}$ with respect to $\mu$, $\mu^{\RV{X}}:\sigalg{X}\to [0,1]$ by $\mu^{\RV{X}}(A):=\mu(\RV{X}^{-1}(A))$ for any $A\in \sigalg{X}$.
\end{definition}

\begin{definition}[Conditional distribution]\label{def:disint}
Given a probability space $(\mu,\Omega,\sigalg{F})$ and variables $\RV{X}:\Omega\to X$, $\RV{Y}:\Omega\to Y$, the \emph{conditional distribution} of $\RV{Y}$ given $\RV{X}$ is any Markov kernel $\mu^{\RV{Y}|\RV{X}}:X\kto Y$ such that
\begin{align}
	\mu^{\RV{XY}}(A\times B)&=\int_{A} \mu^{\RV{Y}|\RV{X}}(B|x) \mathrm{d}\mu^{\RV{X}}(x) &\forall A\in \sigalg{X}, B\in \sigalg{Y}
\end{align}
\end{definition}

\begin{definition}[Trivial variable]\label{no:single_valued}
We let $*$ stand for any single-valued variable $*:\Omega\to \{*\}$.
\end{definition}

% \subsubsection{Markov kernel product notation}\label{ssec:product_notation}

% Three pairwise \emph{product} operations involving Markov kernels can be defined: measure-kernel products, kernel-kernel products and kernel-function products. These are analagous to row vector-matrix products, matrix-matrix products and matrix-column vector products respectively.

% \begin{definition}[Measure-kernel product]
% Given $\mu\in \Delta(\mathcal{X})$ and $\kernel{M}:X\kto Y$, the \emph{measure-kernel product} $\mu\kernel{M}\in \Delta(Y)$ is given by
% \begin{align}
% \mu\kernel{M} (A) := \int_X \kernel{M}(A|x) \mu(\mathrm{d}x)
% \end{align}
% for all $A\in \sigalg{Y}$.
% \end{definition}

% \begin{definition}[Kernel-kernel product]\label{def:kproduct}
% Given $\kernel{M}:X\kto Y$ and $\kernel{N}:Y\kto Z$, the \emph{kernel-kernel product} $\kernel{M}\kernel{N}:X\kto Z$ is given by
% \begin{align}
% \kernel{MN} (A|x) := \int_Y \kernel{N}(A|x) \kernel{M}(\mathrm{d}y|x)
% \end{align}
% for all $A\in \sigalg{Z}$, $x\in X$.
% \end{definition}

% \begin{definition}[Kernel-function product]
% Given $\kernel{M}:X\kto Y$ and $f:Y\to Z$, the \emph{kernel-function product} $\kernel{M}f:X\to Z$ is given by
% \begin{align}
% \kernel{M}f (x) := \int_Y f(y)\kernel{N}(\mathrm{d}y|x)
% \end{align}
% for all $x\in X$.
% \end{definition}

% \begin{definition}[Tensor product]
% Given $\kernel{M}:X\kto Y$ and $\kernel{L}:W\kto Z$, the tensor product $\kernel{M}\otimes\kernel{N}:X\times W\kto Y\times Z$ is given by
% \begin{align}
% 	(\kernel{M}\otimes\kernel{L})(A\times B|x,w):=\kernel{M}(A|x)\kernel{L}(B|w)
% \end{align}
% For all $x\in X$, $w\in W$, $A\in \sigalg{Y}$ and $B\in \sigalg{Z}$.
% \end{definition}

% All products are associative \citep[Chapter 1]{cinlar_probability_2011}.

\subsection{Decision models}\label{sec:probability_sets}

A \emph{decision model} is a Markov kernel $\prob{P}_\cdot$ from an option set $(C,\sigalg{C})$ to a sample space $(\Omega,\sigalg{F})$.

\begin{definition}[Decision model]\label{def:dec_model}
A decision model is a triple $(\prob{P}_\cdot, (\Omega,\sigalg{F}), (C,\sigalg{C}))$ where $\prob{P}_\cdot:C\kto \Omega$ is a Markov kernel, $(\Omega,\sigalg{F})$ is the sample space and $(C,\sigalg{C})$ is the option set.
\end{definition}

For an option $\alpha\in C$, we say $\prob{P}_\alpha$ is the model $\prob{P}_\cdot$ evaluated at $\alpha$.

\begin{definition}[Almost sure equality]
Given a decision model $(\prob{P}_\cdot, (\Omega,\sigalg{F}), (C,\sigalg{C}))$ and random variables $\RV{X}:\Omega\to X$, $\RV{Y}:\Omega\to Y$, two Markov kernels $\kernel{K}:X\kto Y$ and $\kernel{L}:X\kto Y$ are $\prob{P}_\cdot,\RV{X},\RV{Y}$-almost surely equal if for all $A\in\sigalg{X}$, $B\in \sigalg{Y}$, $\alpha\in C$
\begin{align}
    \int_A \kernel{K}(B|x)\prob{P}_\alpha^{\RV{X}}(\mathrm{d}x) = \int_A\kernel{L}(B|x)\prob{P}_\alpha^{\RV{X}}(\mathrm{d}x)
\end{align}
we write this as $\kernel{K}\overset{\prob{P}_\cdot^{\RV{X}}}{\cong}\kernel{L}$.
\end{definition}

Equivalently, $\kernel{K}$ and $\kernel{L}$ are almost surely equal if the set $C:\{x|\exists B\in\sigalg{Y}:\kernel{K}(B|x)\neq\kernel{L}(B|x)\}$ has measure 0 with respect to $\prob{P}_\alpha^{\RV{X}}$ for all $\alpha\in C$.

\subsection{Extended conditional independence}\label{sec:eci}

Because decision models aren't standard probability spaces, we need some version of conditional independence for decision models. Such a notion has already been worked out in some detail: it is the idea of \emph{extended conditional independence} defined in \citet{constantinou_extended_2017}. Extended conditional independence is substantially more general than we need for our purposes, and in fact we only consider two special cases of it. However, we still make use of the notational convention introduced in that paper.

We will first define regular conditional independence. We define it in terms of a having a conditional that ``ignores one of its inputs'', which, provided conditional probabilities exists, is equivalent to other common definitions todo cite

\begin{definition}[Conditional independence]\label{def:ci}
Given a decision model $(\prob{P}_\cdot, (\Omega,\sigalg{F}), (C,\sigalg{C}))$, variables $\RV{X},\RV{Y},\RV{Z}$ and fixing some $\alpha\in C$, we say $\RV{Y}$ is conditionally independent of $\RV{X}$ given $\RV{Z}$, written $\RV{Y}\CI_{\model{P}_{\alpha}}\RV{X}|\RV{Z}$, if there exists some $\kernel{K}:Z\kto Y$ such that
\begin{align}
    \prob{P}^{\RV{Y}|\RV{XZ}}(A|x,z) &\overset{\prob{P}_\alpha^{\RV{XZ}}}{\cong} \prob{K}(A|z)&\forall A\in \sigalg{Y}
\end{align}
\end{definition}

Extended conditional independence as introduced by \citet{constantinou_extended_2017} is defined using ``nonstochastic variables'' on the option set C. For our purposes, it is sufficient to use only the special nonstochastic variable $\mathrm{id}_C:C\to C$.

Our two notions are \emph{global conditional independence} and \emph{uniform conditional independence}. The former can be understood as meaning ``conditional independence for every $\alpha\in C$'', while the latter means ``conditional independence for every $\alpha\in C$ and moreover not dependent on $\alpha$''.

\begin{definition}[Global conditionally independence]\label{def:eci_orig}
Given a decision model $(\prob{P}_\cdot, (\Omega,\sigalg{F}), (C,\sigalg{C}))$ and variables $\RV{X}$, $\RV{Y}$ and $\RV{Z}$, $\RV{Y}$ is globally independent of $\RV{X}$ given $\RV{Z}$, written $\RV{Y}\CI^e_{\prob{P}_\cdot} \RV{X} |(\RV{Z}, \mathrm{id}_C)$ if for each $\alpha\in C$
\begin{align}
    \prob{P}_{\alpha}^{\RV{Y}|\RV{XZ}}(A|x,z) &\overset{\prob{P}_{\alpha}^{\RV{XZ}}}{\cong} \prob{P}_{\alpha}^{\RV{Y}|\RV{Z}}(A|z)&\forall A\in \sigalg{Y},(x,z)\in X\times Z\label{eq:eci}
\end{align}
\end{definition}

\begin{definition}[Uniform conditional independence]\label{def:eci}
Given a decision model $(\prob{P}_\cdot, (\Omega,\sigalg{F}), (C,\sigalg{C}))$ and variables $\RV{X}$, $\RV{Y}$ and $\RV{Z}$, the uniform conditional independence $\RV{Y}\CI^e_{\prob{P}_\cdot} (\RV{X}, \mathrm{id}_C)|\RV{Z}$ holds if $\RV{Y}\CI^e_{\prob{P}_\cdot} \RV{X} |(\RV{Z}, \mathrm{id}_C)$ and furthermore for all $\alpha,\alpha'\in C$
\begin{align}
    \prob{P}_\alpha^{\RV{Y}|\RV{XZ}} &\overset{\prob{P}_\alpha^{\RV{XZ}}}{\cong} \prob{P}_{\alpha'}^{\RV{Y}|\RV{XZ}}\label{eq:uci}
\end{align}
\end{definition}

For countable sets $C$, as shown by \citet{constantinou_extended_2017}, we can reason with collections of extended conditional independence statements as if they were regular conditional independence statements. In the following rules, $\phi$ and $\xi$ refer to complementary variables on the set $C$ (see \citet{constantinou_extended_2017} for details), but for our purposes we only consider the cases where either $\phi=\mathrm{id}_C$ and $\xi=*$ or $\phi=*$ and $\xi=\mathrm{id}_C$, where $*$ is the trivial variable $\cdot \mapsto *$. In the rest of this text, we will omit the trivial variable from extended conditional independence statements.

\begin{enumerate}
    \item Symmetry: $\RV{X}\CI_{\prob{P}_{\cdot}}^e (\RV{Y}, \phi)|(\RV{Z}, \xi)$ iff $\RV{Y}\CI_{\prob{P}_{\cdot}}^e (\RV{X}, \phi)|(\RV{Z},\xi)$
    \item $\RV{X}\CI_{\prob{P}_{\cdot}}^e (\RV{Y}, \mathrm{id}_C)| (\RV{Y}, \mathrm{id}_C)$
    \item Decomposition: $\RV{X}\CI_{\prob{P}_{\cdot}}^e (\RV{Y}, \phi)|\RV{W}\xi$ and $\RV{Z}\varlessthan\RV{Y}$ implies $\RV{X}\CI_{\prob{P}_{\cdot}}^e(\RV{Z},\phi)|(\RV{W},\xi)$
    \item Weak union:
    \begin{enumerate}
     	\item $\RV{X}\CI^e_{\prob{P}_{\cdot}} (\RV{Y}, \phi)|(\RV{W}, \xi)$ and $\RV{Z}\varlessthan \RV{Y}$ implies $\RV{X}\CI_{\prob{P}_{\cdot}}^e(\RV{Y},\phi)|(\RV{Z},\RV{W}, \xi)$
     	\item $\RV{X}\CI_{\prob{P}_{\cdot}}^e \RV{Y} \mathrm{id}_{C}|\RV{W}$ implies $\RV{X}\CI_{\prob{P}_{\cdot}}^e\RV{Y}|(\RV{W},\mathrm{id}_C)$
     \end{enumerate} 
    \item Contraction: $\RV{X}\CI_{\prob{P}_{\cdot}}^e(\RV{Z},phi)|(\RV{W},\xi)$ and $\RV{X}\CI_{\prob{P}_{\cdot}}^e(\RV{Y},\phi)|(\RV{Z},\RV{W})\xi$ implies $\RV{X}\CI_{\prob{P}_{\cdot}}^e(\RV{Y},\RV{Z},\phi)|(\RV{W},\xi)$
\end{enumerate} 

If we have the extended conditional independence $\RV{Y}\CI^e_{\prob{P}_\cdot} \mathrm{id}_C | \RV{X}$, then by definition for all $\alpha,\alpha'\in C$ we have $\prob{P}_\alpha^{\RV{Y}|\RV{X}}=\prob{P}_{\alpha'}^{\RV{Y}|\RV{X}}$. In this case, we use the notation $\prob{P}_C^{\RV{Y}|\RV{X}}$ to indicate that the conditional distribution does not depend on the choice of $\alpha$

\begin{definition}[Uniform conditional distribution]\label{def:uci}
Given a decision model $(\prob{P}_\cdot, (\Omega,\sigalg{F}), (C,\sigalg{C}))$ and variables $\RV{X}$, $\RV{Y}$, if $\RV{Y}\CI^e_{\prob{P}_\cdot} \mathrm{id}_C | \RV{X}$ then
\begin{align}
    \prob{P}_C^{\RV{Y}|\RV{X}} &= \prob{P}_\alpha^{\RV{Y}|\RV{X}}
\end{align}
for any $\alpha\in C$. If $\RV{Y}\not \CI^e_{\prob{P}_\cdot} \mathrm{id}_C | \RV{X}$ then $\prob{P}_C^{\RV{Y}|\RV{X}}$ is not defined.
\end{definition}

